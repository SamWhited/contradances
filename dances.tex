\documentclass[medium]{contracard}

\newcommand*{\definemove}[2]{
	\ignorespaces\textit{#1}\unskip\ — \ignorespaces#2\unskip
}

\usepackage{fontspec}
\newcommand*{\defaultfontfamily}{Calluna}
\newcommand*{\dingbatsfontfamily}{DejaVu Sans}

\defaultfontfeatures{Mapping=tex-text, Ligatures=Common}
\setmainfont[Mapping=tex-text, RawFeature={+lnum}]{\defaultfontfamily}

\newcommand*{\ordn}[1]{{\fontspec[RawFeature={+ordn}]{\defaultfontfamily}#1}}
\newcommand*{\titl}[1]{{\fontspec[RawFeature={+titl}]{\defaultfontfamily}#1}}
\newcommand*{\ornm}[1]{{\fontspec[RawFeature={+ornm}]{\defaultfontfamily}#1}}
\newcommand*{\tnum}[1]{{\fontspec[RawFeature={+tnum}]{\defaultfontfamily}#1}}
\newcommand*{\onum}[1]{{\fontspec[RawFeature={+onum}]{\defaultfontfamily}#1}}
\newcommand*{\lnum}[1]{{\fontspec[RawFeature={+lnum}]{\defaultfontfamily}#1}}

\newcommand*{\dingbat}[1]{{\fontspec{\dingbatsfontfamily}#1}}

\newfontfamily\partfont[Mapping=tex-text, RawFeature={+titl,+onum,+liga,+dlig,+cpsp}]{\defaultfontfamily}

\begin{document}

\part{\partfont Duple becket}

\begin{contra}{Uncommon Courtesy}{Gene Hubert}{Duple becket}
	\move[8]{Star left 1}
	\move[8]{Partner courtesy turn and roll away}
	\move[8]{Pass through across}
	\move[8]{Partner swing}

	\move[6]{Circle left ¾}
	\move[10]{Partner swing}
	\move[8]{Ladies chain}
	\move[8]{Promenade and loop to face new neighbors}
\end{contra}

\begin{contra}{Give the Scout a Hand}{Bob Isaacs \&\ Chris Weiler}{Duple becket}
	\move[2]{Slide left}
	\move[6]{Circle left ¾}
	\move[8]{Neighbor swing}
	\move[8]{Long lines forward \&\ back}
	\move[8]{Gents allemande left 1½ to partner}

	\move[4]{Partner right hand balance}
	\move[4]{box the gnat}
	\move[8]{½ hey with hands}
	\move[16]{Partner balance, and swing}
\end{contra}

\begin{contra}{Troy Lee's Bingo}{John Nance}{Duple becket}
	\move[8]{Blackberry jam}
	\move[8]{Circle left $\times 1$}
	\swing*[Partner]{16}

	\move[8]{Bingo petro}
	\swing*[Neighbor]{8}
	\ladieschain
	\move[8]{Deviled eggs}

	\notes{%
		\definemove{Blackberry jam}{LL forward, hip bump neighbor, back.}\newline%
		\definemove{Bingo petro}{B\&P; shout `bingo' for clap}\newline%
		\definemove{Deviled eggs}{men $\times$, then right diagonal while ladies
		$\times$ left diagonal then across. Shout `Troy Lee!'}%
	}

\end{contra}

\begin{contra}{Carole and Bill Get a Toaster}{Seth Tepfer}{Duple becket}
	\move[6]{Left hand star ¾}
	\swing*[Neighbor]{10}
	\longlines
	\ladieschain*[Left diagonal,]
	\balanceand\longpetronella\balanceand\longpetronella
	\swing*[Turn around and partner]{16}
	\notes{Call the neighbor swing early. Composed for the wedding of Carole
	Galanty and Bill Porter. Inspired by a toaster given to Seth and Pam by Ann
	Robinson and Joseph Steinhauser.}
\end{contra}

\begin{contra}{Code's Compiling}{Sam Whited}{Duple becket}
	\move[8]{Neighbor do-si-do}
	\move[8]{Partner see saw}
	\move[16]{Shadow swing}
	\move[8]{Balance and petronella}
	\move[8]{Balance and petronella}
	\move[8]{Partner swing}
	\move[8]{Left diagonal: Right and left through}
	\notes{Double progression, advanced. Don't need to stop and face neighbor in
	see saw. Call it a mad robin if it makes you feel better. Make sure lines have
  lots of space for do-si-do.}
\end{contra}

\begin{contra}{Monarch Grove}{Martha Wild}{Duple becket}
	\move{Slide left and circle left \textthreequarters\moveindex{Circle Left}\moveindex{Slide Left}}
	\swing[neighbor]{8}
	\longlines
	\dosido*[Men]{8}
	\allemande*[Men]{8}{left 1\textonehalf}
	\move{Partner star promenade and butterfly whirl\moveindex{Star Promenade}\moveindex{Butterfly Whirl}}
	\dosido*[Women]{8}
	\swing[partner]{8}
\end{contra}

\clearpage
\part{\partfont Duple improper}

\begin{contra}{Broken Sixpence}{Don Armstrong}{Duple improper}
	\move[8]{Neighbor do-si-do}
	\move[8]{Gents do-si-do}
	\move[8]{Ladies do-si-do}
	\move[8]{Actives swing}

	\move[8]{Down the hall, turn alone}
	\move[8]{Up the hall, bend the line}
	\move[8]{Circle left 1 time}
	\move[8]{Left hand star to new neighbors}
	
	\notes{Beginner}
\end{contra}

\begin{contra}{Swing-a-Doodle}{Don Armstrong}{Duple improper}
	\move[16]{Partner balance and swing (end facing down)}
	\move[8]{Down in 2, 1's turn as a couple, 2's cast out and up to 1's}
	\move[8]{Up in 4, bend the line}
	\move[16]{Balance and swing neighbor}
	\move[8]{½ promenade across}
	\move[8]{Do-si-do across}
	\notes{Advanced}
\end{contra}

\begin{contra}{Simplicity Swing}{Becky Hill}{Duple improper}
	\move[16]{Neighbor balance and swing}
	\move[8]{Circle left ¾}
	\move[8]{Partner swing}

	\move[8]{Long lines go forward and back}
	\move[8]{Ladies chain}
	\move[8]{Left hand star}
	\move[8]{Next neighbor do-si-do}

	\notes{Beginner}
\end{contra}

\begin{contra}{ABC}{Erik Hoffman}{Duple improper}
	\move[16]{Neighbor balance and swing}
	\move[8]{Right and left through}
	\move[8]{Right and left through back}

	\move[8]{Ladies chain}
	\move[8]{Chain on back}
	\move[8]{Long lines forward and back}
	\move[8]{Actives swing}

	\notes{Beginner. Good dance for teaching actives and 1's and 2's.}
\end{contra}

\begin{contra}{Small Potatoes}{Jim Kitch}{Duple improper}
	\move[16]{Neighbor balance and swing}
	\move[8]{Circle left all the way}
	\move[8]{Ladies chain}
	\move[8]{Ladies do-si-do}
	\move[8]{Partner swing}
	\move[6]{Circle left ¾}
	\move[2]{Pass through}
	\move[8]{Next neighbor do-si-do}
\end{contra}

\begin{contra}{Trip to Atlanta}{John Nance}{Duple improper}
	\allemande*[Neighbor \#1]{8}{right 1½}
	\swing*[Neighbor \#2]{8}
	\promenade{8}
	\ladieschain

	\move[8]{Pass through across}
	\swing*[Partner]{8}
	\move[6]{Circle left ¾}
	\move[2]{Pass through up and down}
	\dosido[New neighbor \#1]{8}

	\notes{Double progression}
\end{contra}

\begin{contra}{Gypsy Road}{Seth Tepfer}{Duple improper}
	\move[16]{Neighbor gypsy and swing}
	\move[8]{Long lines}
	\move[8]{Gents allemande left 1½}
	\move[16]{Partner balance and swing}
	\move[4]{Circle left ½}
	\move[4]{Ladies roll partner away with a ½ sashay}
	\move[3]{Face across, neighbor ½ gypsy}
	\move[2]{Single file prom CW 1 place}
	\move[3]{Pass neighbor right}
	\notes{CC BY-NC 3.0}
\end{contra}

\begin{contra}{Whirling dervish}{Sam Whited}{Duple improper}
	\gypsy[neighbor 1¾]{8}
	\move[8]{Men half hey ricochet  (ladies cross)%
		\moveindex{Hey For Four}%
		\moveindex{Half Hey}%
	}
	\swing[your partners all]{16}
	\move[8]{Spin like a Whirling Dervish\moveindex{Whirling Dervish}}
	\balanceand\longpetronella
	\balanceand\move[8]{gents roll neighbor away with a half sashay\moveindex{Half
	Sashay}\moveindex{Roll Away}\moveindex{Roll Away\ldots Half Sashay}}
	\newline\move[8]{Balance neighbor and gypsy left 1 time to new
	neighbors\moveindex{Balance}\moveindex{Gypsy}\moveindex{Gypsy Left}}
	\notes{A ``Whirling Dervish" is just a circle left in single file except that
	you should spin over your left shoulder the entire time you're doing it.}
\end{contra}

\begin{contra}{Tetrahymena Twirl}{Martha Wild}{Duple improper}
	\move{Handy hand allemande neighbor 1\textonehalf\moveindex{Allemande}}
	\swing[\#2's]{8}\footnote{\#2's are now above \#1's}
	\move{Down set four in line\footnote{The line is man/man/woman/woman, \#2's in
	the middle}, men turn as a couple\footnote{The left hand man should lead the
couple turn to avoid confusion}, women turn alone\moveindex{Down the hall}}
	\move{Come back up, bend the line}

	\move{Balance in a circle, pass through across the set}
	\move{Star right 1\textthreequarters}
	\move{Balance the star, \#2's arch, \#1 man pull partner under arch}
	\newline\swing*[\#1's]{8}
\end{contra}

\begin{contra}{East Meets West}{Martha Wild}{Duple improper}
  \longlines
  \move{Gypsy star \textthreequarters}\moveindex{Gypsy Star}
  \move[16]{Gypsy and swing partner}\moveindex{Gypsy}\moveindex{Swing}
  \halfpromenade
  \move{Hey}\moveindex{Hey}\moveindex{Hey for Four}\moveindex{Full Hey}
  \move{(continue hey)}
  \ladieschain
\end{contra}

% TODO: Identify
\begin{contra}{Unknown contra}{Unknown}{Duple improper}
	\move[8]{Neighbor do-si-do}
	\move[8]{Neighbor swing}
	\move[6]{Circle left ¾}
	\move[10]{Partner swing}
	\move[8]{Long lines forward and back}
	\move[8]{Ladies chain}
	\move[8]{Long lines forward and back}
	\move[8]{Left hand star to new neighbors}
	\notes{Beginner}
\end{contra}

\begin{contra}{Lament to Sarah}{Sam Whited}{Duple improper}
	\move[8]{In your set: short lines forward \&\ back}
	\move[4]{Roll away with a ½ sashay}
	\move[4]{With your neighbor two hand turn}
	\move[8]{(cont. two hand turn)}
	\move[8]{Melt into a swing}
	\balanceand\longpetronella\balanceand\longpetronella
	\balanceand\longpetronella
	\balanceand\move[8]{with your partner two hand turn, while rotating to the
	right to progress}

	\notes{In A1. continue to hold your neighbors hand as you back out of the
	lines and instead of stepping directly backwards away from your neighbor,
	open up to face across in those 4 steps.}
\end{contra}

\part{\partfont Duple minor improper}

\newlength{\phrasevspaceold}
\setlength{\phrasevspaceold}{\phrasevspace}
\setlength{\phrasevspace}{0em}
\setcounter{dancephraselength}{16}
% TODO: Find a font in which the numero sign looks somewhat like Calluna.
\begin{contra}{Untitled Blues Contra \dingbat{№} 1}{Seth Tepfer}{Duple Minor Im.}
	\move[8]{Neighbor gypsy}
	\move[8]{Ladies gypsy}
	\move[16]{Partner balance and swing}
	\move[8]{Ladies chain}
	\move[8]{Left hand star}
	\notes{CC BY-NC 3.0}
\end{contra}

\begin{contra}{Untitled Blues Contra \dingbat{№} 2}{Seth Tepfer}{Duple Minor Im.}
	\move[8]{Neighbor gypsy}
	\move[8]{Gents allemande left 1½}
	\move[16]{Partner balance and swing}
	\move[8]{½ promenade across set}
	\move[8]{Ladies chain}
	\notes{CC BY-NC 3.0}
\end{contra}

\setlength{\phrasevspace}{\phrasevspaceold}
\setcounter{dancephraselength}{32}
\clearpage
\part{\partfont Mescolanza}

\begin{contra}{No More Worthless Swings}{Seth Tepfer}{Mescolanza}
	\move[8]{Long lines}
	\move[8]{Up \&\ down ladies allemande left 1½}
	\move[16]{Neighbor balance and swing}
	\move[4]{Gents left hand star ½ to partner}
	\move[4]{Allemande right ½}
	\move[8]{Ladies left hand star all the way}
	\move[16]{Partner balance \&\ swing}
	\notes{CC BY-NC 3.0}
\end{contra}

\part{\partfont Waltz Mixer}

\begin{contra}{Turn Around Waltz}{Sam Whited}{Circle waltz}
	\setlength{\phrasevspace}{0em}
	\setcounter{dancepartlength}{12}
	\setcounter{dancephraselength}{24}

	\move[6]{Roll lady on left away}
	\move[6]{Forward and back}
	\move[6]{Roll lady on left away}
	\move[6]{Forward and back}
	\move[6]{Hands out and in (face out)}
	\move[6]{Hands in and out (face in)}
	\move[6]{Hands out and in ladies cast right}
	\move[6]{Parallels (or free waltz)}

	\notes{Repeat \textbf{BABB}(\textbf{Y}?). Doing it in circles of 4 or 8
	couples brings you back to your original partner at the end; if you have 3 or
	7 put partner on your left to start. Get people into square formation, then
	ask them to merge squares instead of finding 7 other people. Meant to be
	danced to the Nanci Griffith cover of ``Turn Around".}
\end{contra}
\end{document}
