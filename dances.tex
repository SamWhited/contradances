\documentclass[a7paper]{contracard}

% TODO: Add this to contracard
\newcommand*{\notes}[1]{
	\vspace{\fill}
	\begin{flushleft}
		{\footnotesize \textbf{Notes} #1}
	\end{flushleft}
}

\usepackage{fontspec}
\newcommand*{\defaultfontfamily}{Calluna}
\newcommand*{\dingbatsfontfamily}{DejaVu Sans}

\defaultfontfeatures{Mapping=tex-text, Ligatures=Common}
\setmainfont[Mapping=tex-text]{\defaultfontfamily}

\newcommand*{\ordn}[1]{{\fontspec[RawFeature={+ordn}]{\defaultfontfamily}#1}}
\newcommand*{\titl}[1]{{\fontspec[RawFeature={+titl}]{\defaultfontfamily}#1}}
\newcommand*{\ornm}[1]{{\fontspec[RawFeature={+ornm}]{\defaultfontfamily}#1}}

\newcommand*{\dingbat}[1]{{\fontspec{\dingbatsfontfamily}#1}}

\newfontfamily\partfont[Mapping=tex-text, RawFeature={+titl}]{\defaultfontfamily}

\begin{document}

\part{\partfont Duple becket}

\begin{contra}{Uncommon Courtesy}{Gene Hubert}{Duple becket}
	\move[8]{Star left 1}
	\move[8]{Partner courtesy turn and roll away}
	\move[8]{Pass through across}
	\move[8]{Partner swing}

	\move[6]{Circle left ¾}
	\move[10]{Partner swing}
	\move[8]{Ladies chain}
	\move[8]{Promenade and loop to face new neighbors}
\end{contra}

\begin{contra}{Give the Scout a Hand}{Bob Isaacs \&\ Chris Weiler}{Duple becket}
	\move[2]{Slide left}
	\move[6]{Circle left ¾}
	\move[8]{Neighbor swing}
	\move[8]{Long lines forward \&\ back}
	\move[8]{Gents allemande left 1½ to partner}

	\move[4]{Partner right hand balance}
	\move[4]{box the gnat}
	\move[8]{½ hey with hands}
	\move{Partner balance, and swing}
\end{contra}

\begin{contra}{Code's Compiling}{Sam Whited}{Duple becket}
	\move[8]{Neighbor do-si-do}
	\move[8]{Mad robin}
	\move{Shadow swing}
	\move[8]{Balance and petronella}
	\move[8]{Balance and petronella}
	\move[8]{Partner swing}
	\move[8]{Left diagonal: Right and left through}
	\notes{Double progression, advanced. Don't need to look at neighbor in mad
	robin. See saw works too. Make sure lines have lots of space for do-si-do.}
\end{contra}

\part{\partfont Duple improper}

\begin{contra}{Broken Sixpence}{Don Armstrong}{Duple improper}
	\move[8]{Neighbor do-si-do}
	\move[8]{Gents do-si-do}
	\move[8]{Ladies do-si-do}
	\move[8]{Actives swing}

	\move[8]{Down the hall, turn alone}
	\move[8]{Up the hall, bend the line}
	\move[8]{Circle left 1 time}
	\move[8]{Left hand star to new neighbors}
	
	\notes{Beginner}
\end{contra}

\begin{contra}{Swing-a-Doodle}{Don Armstrong}{Duple improper}
	\move{Partner balance and swing (end facing down)}
	\move[8]{Down in 2, 1's turn as a couple, 2's cast out and up to 1's}
	% TODO: Fix hanging indents in contracard package
	\move[8]{\hspace*{\fill}\\*\hspace*{1.5em}Up in 4, bend the line}
	\move{Balance and swing neighbor}
	\move[8]{½ promenade across}
	\move[8]{Do-si-do across}
	\notes{Advanced}
\end{contra}

\begin{contra}{Simplicity Swing}{Becky Hill}{Duple improper}
	\move{Neighbor balance and swing}
	\move[8]{Circle left ¾}
	\move[8]{Partner swing}

	\move[8]{Long lines go forward and back}
	\move[8]{Ladies chain}
	\move[8]{Left hand star}
	\move[8]{Next neighbor do-si-do}

	\notes{Beginner}
\end{contra}

\begin{contra}{ABC}{Erik Hoffman}{Duple improper}
	\move{Neighbor balance and swing}
	\move[8]{Right and left through}
	\move[8]{Right and left through back}

	\move[8]{Ladies chain}
	\move[8]{Chain on back}
	\move[8]{Long lines forward and back}
	\move[8]{Actives swing}

	\notes{Beginner. Good dance for teaching actives and 1's and 2's.}
\end{contra}

\begin{contra}{Small Potatoes}{Jim Kitch}{Duple improper}
	\move{Neighbor balance and swing}
	\move[8]{Circle left all the way}
	\move[8]{Ladies chain}
	\move[8]{Ladies do-si-do}
	\move[8]{Partner swing}
	\move[6]{Circle left ¾}
	\move[2]{Pass through}
	\move[8]{Next neighbor do-si-do}
\end{contra}

\begin{contra}{Gypsy Road}{Seth Tepfer}{Duple improper}
	\move{Neighbor gypsy and swing}
	\move[8]{Long lines}
	\move[8]{Gents allemande left 1½}
	\move{Partner balance and swing}
	\move[4]{Circle left ½}
	\move[4]{Ladies roll partner away with a ½ sashay}
	% TODO: Go look this one up and fix formatting on this bit
	\move[3]{Face across, neighbor ½ gypsy}
	\move[2]{Single file prom CW 1 place}
	\move[3]{Pass neighbor right}
	%\notes{CC BY-NC 3.0}
\end{contra}

% TODO: Identify
\begin{contra}{Unknown contra}{Unknown}{Duple improper}
	\move[8]{Neighbor do-si-do}
	\move[8]{Neighbor swing}
	\move[6]{Circle left ¾}
	\move[10]{Partner swing}
	\move[8]{Long lines forward and back}
	\move[8]{Ladies chain}
	\move[8]{Long lines forward and back}
	\move[8]{Left hand star to new neighbors}
	\notes{Beginner}
\end{contra}

\part{\partfont Duple minor improper}

\newlength{\phrasevspaceold}
\setlength{\phrasevspaceold}{\phrasevspace}
\setlength{\phrasevspace}{0em}
\setcounter{dancephraselength}{16}
% TODO: Find a font in which the numero sign looks somewhat like Calluna.
\begin{contra}{Untitled Blues Contra \dingbat{№} 1}{Seth Tepfer}{Duple Minor Im.}
	\move[8]{Neighbor gypsy}
	\move[8]{Ladies gypsy}
	\move[16]{Partner balance and swing}
	\move[8]{Ladies chain}
	\move[8]{Left hand star}
	\notes{CC BY-NC 3.0}
\end{contra}

\begin{contra}{Untitled Blues Contra \dingbat{№} 2}{Seth Tepfer}{Duple Minor Im.}
	\move[8]{Neighbor gypsy}
	\move[8]{Gents allemande left 1½}
	\move[16]{Partner balance and swing}
	\move[8]{½ promenade across set}
	\move[8]{Ladies chain}
	\notes{CC BY-NC 3.0}
\end{contra}
\setlength{\phrasevspace}{\phrasevspaceold}
\setcounter{dancephraselength}{32}

\part{\partfont Mescolanza}

\begin{contra}{No More Worthless Swings}{Seth Tepfer}{Mescolanza}
	\move[8]{Long lines}
	\move[8]{Up \&\ down ladies allemande left 1½}
	\move{Neighbor balance and swing}
	\move[4]{Gents left hand star ½ to partner}
	\move[4]{Allemande right ½}
	\move[8]{Ladies left hand star all the way}
	\move{Partner balance \&\ swing}
	\notes{CC BY-NC 3.0}
\end{contra}

\end{document}
